\documentclass[a4paper]{article}
\usepackage[T1]{fontenc}
\usepackage[utf8]{inputenc}
\usepackage{lmodern}
\usepackage[english]{babel}
\begin{document}

\section{Was ist MEP?}
MEP steht für "Magento Export Plattform" und soll das exportieren von
Produktezu verschieden Preisvergleichsplattformen erleichtern. Dabei 
handelt es sich um ein Neuentwicklung der Firma Flagbit GmbH \& Co. KG 
und basiert auf dem Import-Export Module der Magento Community Edition.

\section{Wie erfolgt die Installation?}
Es werden einmal die Installation über die Quellen bei GitHub mittels 
modman, sowie über den Downloadmanager (//TODO Bild hinzufügen) in 
Magento-Backend unterstützt.  Nach erfolgreichen An- und Abmelden gibt 
es einen neuen Menüpunkt names "MEP".

\section{Wie erstelle ich ein neues Profil?}
In MEP wird jeder neue Export als Profil angesehen. Jedes Profil ist 
eine abgeschlossene Einheit. Das heißt die Profile teilen sich keine 
Daten und keine Konfiguration.

\subsection{Profil erstelen}
Um ein neues Propfil zu erstellen klickt man im Menüpunkt MEP auf 
Export (//TODO Bild hinzufügen). Es wird nun eine Liste der verfügbaren 
Profile angezeigt. Über "Add new" können neue Profile angelegt 
werden. Dazu müssen folgende Daten bereit gestellt werden.

1. "Profile Name": Der Name des Profil
2. "Status": Ein Profil kann aktiv oder inaktive sein. Dies hat 
Auswirkungen beim regelmäßigen, automatischen Export.
3. "Template": In zukünftigen Versionen wird es möglich sein Vorlagen 
für bestimmte Preissuchmaschine zu wählen und damit das Profil zu 
erstellen


(//TODO Bild einfügen vom Profil)


Ein Klick auf "Save and Continue Edit" führt uns in den 
Bearbeitungsmodus eines Profils. Auf der linken Seite sind die 
veschiedene Tabs zu sehen. Diese gruppieren die Eigenschaften des 
Profils. In der rechten oberen Bereich sind die Aktionen zum 
speichern (Save), speichern und weiterbearbeiten (Save and Continue 
Edit), Änderungen verwerfen (Reset) und löschen (Delete) zu sehen. 
Der Aktionsknopf "un" kann zum testen des Profils verwendet werden. 

Achtung: Bei großen Datenmengen kann dies längere Zeit in Anspruch 
nehmen. Daher empfehlen wir den Export zum testen mittels Filter 
einzuschränken. Dazu später mehr.

\subsubsection{Tab Field Mapping}
Im Tab "Field Mapping" werden die Spalten des Exports erstellt. Dazu 
werden die Attribute des Produkte zugeordnet zu den Spalten im Export. 
So kann zum Beispiel die SKU eines Produkte aus Magento in die Feld 
"Artikelnummer" im Export kopiert werden. Dabei ist es möglich das ein 
Attribut mehrfach verwendet werden kann. Zum Beispiel kann die 
description in das Feld "Beschreibung" und gleichzeitig in das Feld 
"Kurzbeschreibung" kopiert werden. Hierbei werden dann zwei Mappings 
nagelegt. Zusehen ist das im folgenden Bild:


(//TODO Bild mit doppelten Mapping einfügen)


Um ein Mapping hinzuzufügen klickt man auf den Aktionsknopf "Add 
Attribute". Daraufhin öffnet sich eine Maske in der folgende Werte 
abgefragt werden:

1. "In Database": Hier werden alle Attribute nach Gruppen aufgelistet. 
Das ist die Quelle für den Wert einer Zeile. 


2. "To Field": Der Name der Spalte in Export


3. "Format": Wird bei festen Werten verwendet; siehe Spezialfälle


4. "Position": Bestimmt die Reihenfolge der Spalten im Export


(//TODO Bild von Add Attrbite mit ausgefüllten Werten)


Ein erneuter Klick auf die Zeile öffnet den Dialog nochmals und es 
können die Werte editiert werden. Mappings können über den "Delete" 
Link am Ende einer jeden Splate in der Tabelle gelöscht werden. Wenn 
mehrere Mappings gleichzeit gelöscht werden sollen, dann können diese 
Markiert und über die Werkzeugleiste über der Tabelle gelöscht werden.


(//TODO Bild von den Massactions)


\paragraph{Spezialfälle}
In der Liste der Attribute gibt es einige Sonderfälle und 
Pseudoattribute.

1. "fixed\_value\_format": In Kombination mit Format können damit feste 
Werte in den Export geschrieben werden. Anwendungsfälle sind hier 
Versandkosten und ähnliches


2. "url": Ist ein Deeplink auf das Produkt, enthält also keine 
zusätzlichen Parameter oder Kategorien


3. "category": Hiermit wird ein Kategoriepfad ähnlich den Breadcrumbs 
dem Export hinzugefügt. Wichtig ist dabei das Kategorietrennzeichen im 
Tab "Dataformat"

\end{document}
