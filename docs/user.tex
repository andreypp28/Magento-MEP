\documentclass[a4paper]{article}
\usepackage[T1]{fontenc}
\usepackage[utf8]{inputenc}
\usepackage{lmodern}
\usepackage[english]{babel}
\begin{document}

\section{Was ist MEP?}
MEP steht für "Magento Export Plattform" und soll das exportieren von
Produktezu verschieden Preisvergleichsplattformen erleichtern. Dabei 
handelt es sich um ein Neuentwicklung der Firma Flagbit GmbH \& Co. KG 
und basiert auf dem Import-Export Module der Magento Community Edition.

\section{Wie erfolgt die Installation?}
Es werden einmal die Installation über die Quellen bei GitHub mittels 
modman, sowie über den Magento Connect Manager (//TODO Bild hinzufügen) in
Magento-Backend unterstützt.  Nach erfolgreichen An- und Abmelden gibt 
es einen neuen Menüpunkt names "MEP".

\section{Wie erstelle ich ein neues Profil?}
In MEP wird jeder neue Export als Profil angesehen. Jedes Profil ist 
eine abgeschlossene Einheit. Das heißt die Profile teilen sich keine 
Daten und keine Konfiguration.

\subsection{Profil erstelen}
Um ein neues Propfil zu erstellen klickt man im Menüpunkt MEP auf 
Export (//TODO Bild hinzufügen). Es wird nun eine Liste der verfügbaren 
Profile angezeigt. Über "Add new" können neue Profile angelegt 
werden. Dazu müssen folgende Daten bereit gestellt werden.

1. "Profile Name": Der Name des Profil
2. "Status": Ein Profil kann aktiv oder inaktive sein. Dies hat 
Auswirkungen beim regelmäßigen, automatischen Export.
3. "Template": In zukünftigen Versionen wird es möglich sein Vorlagen 
für bestimmte Preissuchmaschine zu wählen und damit das Profil zu 
erstellen


(//TODO Bild einfügen vom Profil)


Ein Klick auf "Save and Continue Edit" führt uns in den 
Bearbeitungsmodus eines Profils. Auf der linken Seite sind die 
veschiedene Tabs zu sehen. Diese gruppieren die Eigenschaften des 
Profils. In der rechten oberen Bereich sind die Aktionen zum 
speichern (Save), speichern und weiterbearbeiten (Save and Continue 
Edit), Änderungen verwerfen (Reset) und löschen (Delete) zu sehen. 
Der Aktionsknopf "un" kann zum testen des Profils verwendet werden. 

Achtung: Bei großen Datenmengen kann dies längere Zeit in Anspruch 
nehmen. Daher empfehlen wir den Export zum testen mittels Filter 
einzuschränken. Dazu später mehr.

\subsubsection{Tab Field Mapping}
Im Tab "Field Mapping" werden die Spalten des Exports erstellt. Dazu 
werden die Attribute des Produkte zugeordnet zu den Spalten im Export. 
So kann zum Beispiel die SKU eines Produkte aus Magento in die Feld 
"Artikelnummer" im Export kopiert werden. Dabei ist es möglich das ein 
Attribut mehrfach verwendet werden kann. Zum Beispiel kann die 
description in das Feld "Beschreibung" und gleichzeitig in das Feld 
"Kurzbeschreibung" kopiert werden. Hierbei werden dann zwei Mappings 
nagelegt. Zusehen ist das im folgenden Bild:


(//TODO Bild mit doppelten Mapping einfügen)


Um ein Mapping hinzuzufügen klickt man auf den Aktionsknopf "Add 
Attribute". Daraufhin öffnet sich eine Maske in der folgende Werte 
abgefragt werden:

1. "In Database": Hier werden alle Attribute nach Gruppen aufgelistet. 
Das ist die Quelle für den Wert einer Zeile. 


2. "To Field": Der Name der Spalte in Export. Achtung: Darf nicht mit 
einer Zahl anfangen!


3. "Format": Wird bei festen Werten verwendet; siehe Spezialfälle


4. "Position": Bestimmt die Reihenfolge der Spalten im Export


(//TODO Bild von Add Attrbite mit ausgefüllten Werten)


Ein erneuter Klick auf die Zeile öffnet den Dialog nochmals und es 
können die Werte editiert werden. Mappings können über den "Delete" 
Link am Ende einer jeden Splate in der Tabelle gelöscht werden. Wenn 
mehrere Mappings gleichzeit gelöscht werden sollen, dann können diese 
Markiert und über die Werkzeugleiste über der Tabelle gelöscht werden.


(//TODO Bild von den Massactions)


\paragraph{Spezialfälle}
In der Liste der Attribute gibt es einige Sonderfälle und 
Pseudoattribute.

1. "fixed\_value\_format": In Kombination mit Format können damit feste 
Werte in den Export geschrieben werden. Anwendungsfälle sind hier 
Versandkosten und ähnliches


2. "url": Ist ein Deeplink auf das Produkt, enthält also keine 
zusätzlichen Parameter oder Kategorien


3. "category": Hiermit wird ein Kategoriepfad ähnlich den Breadcrumbs 
dem Export hinzugefügt. Wichtig ist dabei das Kategorietrennzeichen im 
Tab "Dataformat"

\subsubsection{Data Format}
In diesem Tab werden alle Einstellungen für das Format des Exports 
vorgenommen. Das Zielformat ist im Moment auf den Typ CSV beschränkt. 
Formate wie XML und ähnliches sind für zukünftige Versionen in 
Planung.


1. "Value Delimiter": Der Trenner zwischen den Spalten
2. "Enclose values in": Mit welchem Zeichen sollen die Werte eines 
Feldes umschlossen werden. Normalerweise wird " verwendet.
3. "Skip Headers": Soll die Kopfzeile mit den Spaltennamen ausgegeben 
werden oder nicht.
4. "Name of the file": Der Name der Datei für den Export
5. "Path to export": Ich welchen Ordner soll der Export angelegt 
werden. Achtung: Der Ordner muss beschreibar auf dem Server sein. 
Hierbei eignen sich die Ordner media oder var
6. "Separator between categories": Welcher Trenner soll für den 
Kategoriepfad eines Produktes verwendet werden.
7. "Change default locale": Hier kann die Locale Einstellung des 
Servers überschrieben werden. Diese wirkt sich auf das Zahlenformat 
aus. Also ob ein Punkt oder ein Komma als Dezimaltrenner verwendet 
wird. Welche Werte verwendet werden können, hängt von der 
Serverinstallation ab. Bitte fragen sie dies beim Hoster nach.


\paragraph{Spezialfall: Use Templates}
Das CSV-System von MEP basiert auf der Templatesprache Twig. Aus den 
Fieldmappings und den Data Format Optionen wird ein Template für den 
Header und für ein Zeile des Exports generiert. Wenn die Option "Use 
Templates" aktiviert wird, können die Templates direkt bearbeitet 
werden. Dazu muss die Option auf Yes stehen und das Profil einmal 
gespeichert werden. Erst dann steht ein zusätzlicher Tab zur Verfügung.

Achtung: Änderungen an den Fieldmappings müssen dann maunell 
übertragen werden!

\subsubsection{Export Filters}
Jeder Profil in MEP hat die Möglichkeit die Menge der zu exportierden 
Produkte einzusschränken. Dies kann nützlich sein um zum Beispiel nur 
Produkte ab einem bestimmten Preis zu übermitteln oder 
Speziallprodukte (Grantieverlängerungen, Versicherungem, etc.) 
auszuschließen. Technisch basieren die Filter auf den 
Katalogpreisregeln und lassen sich analog verwenden.

\paragraph{Filter anlegen}
Dazu klicken sie auf das grüne Plus-Zeichen und wählen sie das 
Attribut auf welches gefiltert werden soll aus. Alle fett 
unterstrichenen Textelemente sind interaktiv. Sie können den Vergleich 
und den Zielwert bestimmen.

Wenn sie die Option "Conditions Combination" wählen können sie 
verschachtelte Bedingungen erzeugen. Zum Bespiel: "Exportiere alle 
Produkte die einen Preis größer 100€ haben UND einen Lagerbestand 
größer 10 aber unter 50 haben. Im nachfolgenden Bild ist das einmal 
dargestellt.

(//TODO: Bild mit dieser Condtion einfügen)

\paragraph{Filter löschen}

Empfehlung: Bei der Erstellung eines Profils sollten sie die Menge auf 
wenige Produkte beschränken. Wenn der Export korrekt angelegt wurde, 
können sie die Filter wieder entfernen.

\end{document}
