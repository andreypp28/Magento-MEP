\documentclass[a4paper]{article}
\usepackage[T1]{fontenc}
\usepackage[utf8]{inputenc}
\usepackage{lmodern}
\usepackage[english]{babel}
\begin{document}

\section{Was ist MEP?}
MEP steht für "Magento Export Plattform" und soll das exportieren von
Produktezu verschieden Preisvergleichsplattformen erleichtern. Dabei 
handelt es sich um ein Neuentwicklung der Firma Flagbit GmbH \& Co. KG 
und basiert auf dem Import-Export Module der Magento Community Edition.

\section{Wie erfolgt die Installation?}
Es werden einmal die Installation über die Quellen bei GitHub mittels 
modman, sowie über den Downloadmanager (//TODO Bild hinzufügen) in 
Magento-Backend unterstützt.  Nach erfolgreichen An- und Abmelden gibt 
es einen neuen Menüpunkt names "MEP".

\section{Wie erstelle ich ein neues Profil?}
In MEP wird jeder neue Export als Profil angesehen. Jedes Profil ist 
eine abgeschlossene Einheit. Das heißt die Profile teilen sich keine 
Daten und keine Konfiguration.

\subsection{Profil erstelen}
Um ein neues Propfil zu erstellen klickt man im Menüpunkt MEP auf 
Export (//TODO Bild hinzufügen). Es wird nun eine Liste der verfügbaren 
Profile angezeigt. Über "Add new" können neue Profile angelegt 
werden. Dazu müssen folgende Daten bereit gestellt werden.

1. "Profile Name": Der Name des Profil
2. "Status": Ein Profil kann aktiv oder inaktive sein. Dies hat 
Auswirkungen beim regelmäßigen, automatischen Export.
3. "Template": In zukünftigen Versionen wird es möglich sein Vorlagen 
für bestimmte Preissuchmaschine zu wählen und damit das Profil zu 
erstellen

(//TODO Bild einfügen vom Profil)

Ein Klick auf "Save and Continue Edit" führt uns in den 
Bearbeitungsmodus eines Profils. Auf der linken Seite sind die 
veschiedene Tabs zu sehen. Diese gruppieren die Eigenschaften des 
Profils. In der rechten oberen Bereich sind die Aktionen zum 
speichern (Save), speichern und weiterbearbeiten (Save and Continue 
Edit), Änderungen verwerfen (Reset) und löschen (Delete) zu sehen. 
Der Aktionsknopf "un" kann zum testen des Profils verwendet werden. 

Achtung: Bei großen Datenmengen kann dies längere Zeit in Anspruch 
nehmen. Daher empfehlen wir den Export zum testen mittels Filter 
einzuschränken. Dazu später mehr.

\end{document}
